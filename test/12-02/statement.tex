\documentclass[UTF8]{ctexart}
\usepackage{geometry}
\usepackage{fancyhdr}
\usepackage{graphicx}
\usepackage{array}
\usepackage{amsmath}
\usepackage{listings}
\usepackage{xcolor}
\usepackage{fontspec}
\newcommand{\PreserveBackslash}[1]{\let\temp=\\#1\let\\=\temp}
\newcolumntype{C}[1]{>{\PreserveBackslash\centering}p{#1}}
\newcolumntype{R}[1]{>{\PreserveBackslash\raggedleft}p{#1}}
\newcolumntype{L}[1]{>{\PreserveBackslash\raggedright}p{#1}}
\author{p\_b\_p\_b}
\date{day2}
\title{\zihao{2}\textbf{NOIP 2020 模拟赛}}
\geometry{left=3.18cm,right=3.18cm,top=2.5cm,bottom=2.5cm}
\lstset{
	columns=fixed,
	numbers=left,
	stepnumber=3,
	numbersep=5pt,
	numberstyle=\tiny\color{gray},
	frame=none,
	backgroundcolor=\color[RGB]{245,245,244},
	keywordstyle=\color[RGB]{40,40,255},
	numberstyle=\footnotesize\color{darkgray},           
	commentstyle=\it\color[RGB]{0,96,96},
	stringstyle=\rmfamily\slshape\color[RGB]{128,0,0},
	showstringspaces=false,
	language=c++,
}
\begin{document}
	\maketitle
	\setcounter{page}{1}
	\pagenumbering{arabic}
	\textbf{一、题目概况}
	\begin{center}
		\begin{tabular}{*{5}{|L{7em}}|}
			\hline
			题目名称 & 马卡龙 & 长者 & 独白 & 吃瓜\\ \hline
			题目类型 & 传统型 & 传统型 & 传统型 & 传统型\\ \hline
			可执行文件名 & \texttt{macaron} & \texttt{genius} & \texttt{monologue} & \texttt{gourds}\\ \hline
			输入文件名 & \texttt{macaron.in} & \texttt{genius.in} & \texttt{monologue.in} & \texttt{gourds.in}\\ \hline
			输出文件名 & \texttt{macaron.out} & \texttt{genius.out} & \texttt{monologue.out} & \texttt{gourds.out} \\ \hline
			每个测试点时限 & 1.5 秒 & 1.0 秒 & 1.0 秒 & 1.0 秒\\ \hline
			内存限制 & 512 MiB & 256 MiB & 256 MiB & 256 MiB \\ \hline
			子任务数目 & 6 & 2 & 5 & 7 \\ \hline
			测试点是否等分 & 否 & 否 & 否 & 否 \\ \hline
		\end{tabular}
	\end{center}
	\textbf{二、提交源程序文件名}
	\begin{center}
		\begin{tabular}{*{5}{|L{7em}}|}
			\hline
			对于C++语言 & \texttt{macaron.cpp} & \texttt{genius.cpp} & \texttt{monologue.cpp} & \texttt{gourds.cpp}\\ \hline
			对于C语言 & \texttt{macaron.c} & \texttt{genius.c} & \texttt{monologue.c} & \texttt{gourds.c}\\ \hline
		\end{tabular}
	\end{center}
	\textbf{三、编译选项}
	\begin{center}
	\begin{tabular}{|L{7em}|C{31.4em}|}
	\hline
	    对于C++语言 & \texttt{-lm -O2 -std=c++17} \\ \hline
	    对于C语言 & \texttt{-lm -O2} \\ \hline
	\end{tabular}
	\end{center}
	\textbf{四、注意事项}
	\begin{enumerate}
		\item{文件名(程序名和输入输出文件名)必须使用英文小写。}
		\item{C/C++中函数main()的返回类型必须是int,程序正常结束时的返回值必须是0。}
		\item{若无特殊说明,结果的比较方式为全文比较(过滤行末空格及文末回车)}
		\item{程序可使用的栈内存空间限制与题目的内存限制一致。}
		\item{评测时采用的机器配置为:Intel® Core™ i5-9500 CPU @ 3.00GHz ,内存8G,操作系统 Ubuntu 20.04,上述时限和各语言的编译器版本以此为准。}
		\item{\textbf{难度顺序与题目顺序无关。}}
		\item{$\sum$ 是求和运算符,$\sum\limits_{i=1}^na_i$ 的值等于 $a_1+a_2+\dots+a_n$。}
		\item{豫阳市第三工程组提醒您:\textbf{做题千万条,读题第一条;编程不规范,爆零两行泪。}}
	\end{enumerate}
	\newpage
	\pagestyle{plain}
	% T1
	\setcounter{page}{2}
	\newpage
	\section*{\zihao{-1}马卡龙 (\texttt{macaron})}
	\subsection*{【题目背景】}
	马卡龙,又称作玛卡龙、法式小圆饼,是一种用蛋白或aquafaba、糖粉、蔗糖、扁桃仁粉以及食用色素制成的以蛋白脆饼为基础的法式甜点,通常在两块饼干之间夹有甘纳许、奶油乳酪或果酱等内馅。名称源于意大利语单词macarone、maccarone或maccherone,一种意大利蛋白脆饼。————维基百科
	\subsection*{【题目描述】}
	小 K 非常喜欢吃马卡龙。
	
	某天,长者送给了他 $n^2$ 个马卡龙,小 K 把这些马卡龙摆成了 $n\times n$ 的矩阵,第 $i$ 行第 $j$ 列的马卡龙的美味度是 $M_{i,j}$。
	
	他一口气吃不下 $n^2$ 个马卡龙,于是他想先吃 $n$ 个试试口味。为了保持美观,每行每列都只能至多吃掉一个马卡龙。
	
	吃掉 $n$ 个马卡龙后,小 K 的快乐度是这些马卡龙的美味度的异或和。
	
	小 K 想要知道他的快乐度可能是哪些值,由于他忙着吃东西,就把计算的任务丢给了你。
	\subsection*{【输入格式】}
	第一行,一个正整数 $n$。
	
	第 $i$ 行($2\le i\le n+1$),$n$ 个自然数 $M_{i-1,j}$。
	\subsection*{【输出格式】}
	一行,一堆自然数,表示他的快乐度可能是哪些值。
	
	由于小 K 不想写 spj,你需要升序输出这些数。
	\subsection*{【样例 1 输入】}
	\begin{lstlisting}
3
5 9 15
19 7 2
1 0 0
	\end{lstlisting}
	\subsection*{【样例 1 输出】}
	\begin{lstlisting}
2 7 9 10 26 28
	\end{lstlisting}
	\subsection*{【数据范围】}
	\textbf{本题捆绑测试并子任务依赖。}
	
	对于所有数据,$n\le 60,M_{i,j}<2^{12}$。
	\begin{itemize}
		\item{Subtask 1 (8 分): $n\le 10$。}
		\item{Subtask 2 (15 分): $n\le 14,M_{i,j}<2^{10}$。}
		\item{Subtask 3 (15 分): $n\le 18$,依赖 Subtask 1,2。}
		\item{Subtask 4 (35 分): $n\le 40$,依赖 Subtask 3。}
		\item{Subtask 5 (7 分): $M_{i,j}<2$。}
		\item{Subtask 6 (20 分): 无特殊限制,依赖 Subtask 4,5。}
	\end{itemize}
	% T2
	\newpage
	\section*{\zihao{-1}长者 (\texttt{genius})}
	\subsection*{【题目描述】}
	\textbf{本题的编号方式为 0-based(从 $0$ 开始)}。
	
	某一天,长者拿到了一张图,是 $n$ 个点 $m$ 条边的无向简单图 $(V,E)$,第 $i$ 个点有实数权值 $c_i$。长者想要知道是否存在图中点的非空序列 $p_0,p_1,\dots,p_{k-1}$,使得
	\begin{enumerate}
		\item{对于任意 $0\le i<j<k$,有 $p_i\ne p_j$。}
		\item{对于任意 $i\in[0,k)$,有 $(p_i,p_{(i+1)\bmod k})\in E$。}
		\item{$\sum\limits_{i=0}^{k-1}c_{p_i}>\lfloor\frac k2\rfloor$。}
	\end{enumerate}
	\qquad 如果存在这样的序列,TA 希望你按顺序输出任意一个合法的序列,否则你只需要告诉 TA 无解即可。
	
	由于长者见多识广,你需要在一个测试点中处理 $T$ 组数据的询问。
	\subsection*{【输入格式】}
	从输入文件 \texttt{genius.in} 中读入数据。
	
	第一行,一个正整数 $T$,表示数据组数。之后对于每组数据:
	\begin{itemize}
		\item{第一行,两个自然数 $n,m$,表示图的点数和边数。}
		\item{第二行,$n$ 个实数 $c_i$,表示点的权值。}
		\item{之后 $m$ 行,每行两个自然数 $u,v$,表示图的边。}
	\end{itemize}
	\subsection*{【输出格式】}
	输出到文件 \texttt{genius.out} 中。
	
	对于每组数据,一行,若无解则输出 $0$,否则输出序列长度 $k$ 和 $k$ 个自然数 $p_i$。
	\subsection*{【样例 1 输入】}
	\begin{lstlisting}
3
4 4
0.5 0.5 0.5 0.5
0 1
1 2
2 3
3 0
6 7
0.5 0.5 0.334 0.5 0.333 0.334
0 1
0 2
1 3
2 3
1 4
2 5
4 5
5 5
1.0 0.0 0.667 0.5 0.0
0 1
1 2
2 3
3 4
4 1
	\end{lstlisting}
	\subsection*{【样例 1 输出】}
	\begin{lstlisting}
0
5 4 5 2 0 1
2 3 2
	\end{lstlisting}
	\subsection*{【样例 2】}
	见选手目录下的 \texttt{ex\_genius2.in} 和 \texttt{ex\_genius2.out}。
	\subsection*{【数据范围】}
	\textbf{本题的比较方式是自定义校验器 (special judge),忽略行末空格和文末多余字符。}
	
	\textbf{本题捆绑测试并子任务依赖。}
	
	对于所有数据,$T\le 30,1\le n\le 500,0\le m\le 2000,0\le c_i\le 1,10^3\cdot c_i$ 是整数。
	\begin{itemize}
		\item{Subtask 1 (10 分):给定的图是树。}
		\item{Subtask 2 (90 分):无特殊限制,依赖 Subtask 1。}
	\end{itemize}
	% T3
	\newpage
	\section*{\zihao{-1}独白 (\texttt{monologue})}
	\subsection*{【题目描述】}
	\textbf{本题的编号方式为 0-based(从 $0$ 开始)}。
	
	小 K 的独白是一串无限正整数序列,开头 $n$ 个数已被长者钦定为 $a_i$。对于 $i\ge n$ 的情况,
	$$
	a_i=\sum_{j=0}^{i-1}[j+a_j\ge i]
	$$
	\qquad $q$ 次询问,每次给出一个自然数 $k$,求 $a_k$。
	\subsection*{【输入格式】}
	从输入文件 \texttt{monologue.in} 中读入数据。
	
	第一行,两个正整数 $n,q$。
	
	第二行,$n$ 个正整数 $a_i$。
	
	之后 $q$ 行,每行一个自然数 $k$,表示询问。
	\subsection*{【输出格式】}
	输出到文件 \texttt{monologue.out} 中。
	
	$q$ 行,每行一个正整数,表示答案。
	\subsection*{【样例 1-3 输入】}
	见选手目录下的 \texttt{ex\_monologue?.in} 和 \texttt{ex\_monologue?.out}。
	\subsection*{【数据范围】}
	\textbf{本题捆绑测试并子任务依赖。}
	
	对于所有数据,$n,q\le 10^5,k\le 10^{15}$。
	
	\textbf{对任意 $k\in[0,n)$,有 $|a_k-n|\le 1$。}
	\begin{itemize}
		\item{Subtask 1 (10 分):$n,q,k\le 10^3$。}
		\item{Subtask 2 (10 分):$k\le 10^5$,依赖 Subtask 1。}
		\item{Subtask 3 (20 分):$k<n^2$。}
		\item{Subtask 4 (20 分):对任意 $k\in[0,n)$,有 $a_k\ne n$。}
		\item{Subtask 5 (40 分):无特殊限制,依赖 Subtask 2,3,4。}
	\end{itemize}
	% T4
	\newpage
	\section*{\zihao{-1}吃瓜 (\texttt{gourds})}
	\subsection*{【题目描述】}
	又到了瓜果丰收的季节。Yazid、Zayid 和小 K 在分瓜吃。
	
	他们有 $n$ 个瓜,由于 $3|n$,于是他们想平分这些瓜。这些瓜被按顺序摆放成一列,他们对每个瓜的喜好程度可以用三个长为 $n$ 的排列 $a,b,c$ 表示出来,Yazid、Zayid 和小 K 对第 $i$ 个瓜的喜好程度分别为 $a_i,b_i$ 和 $c_i$。
	
	现在他们打算这样分配瓜:做 $\frac n3$ 次操作,每次操作他们分别选出之前还未选过的瓜中最喜欢的瓜,如果出现重复则他们会开始打架,分配就失败了。若自始至终都没有人打架,分配就成功了。
	
	你知道了 $a,b$,求有多少排列 $c$ 使得分配成功。由于答案可能很大,你只需要输出答案对 $10^9+7$ 取模的值。
	\subsection*{【输入格式】}
	从输入文件 \texttt{gourds.in} 中读入数据。
	
	第一行,一个正整数 $n$。
	
	第二行,$n$ 个正整数 $a_i$。
	
	第三行,$n$ 个正整数 $b_i$。
	\subsection*{【输出格式】}
	输出到文件 \texttt{gourds.out} 中。
	
	一行,一个整数,表示答案对 $10^9+7$ 取模的值。
	\subsection*{【样例 1 输入】}
	\begin{lstlisting}
3
1 2 3
2 3 1
	\end{lstlisting}
	\subsection*{【样例 1 输出】}
	\begin{lstlisting}
2
	\end{lstlisting}
	\subsection*{【样例 1 解释】}
	可行的排列 $c$ 有 $(3,1,2)$ 和 $(3,2,1)$。
	\subsection*{【数据范围】}
	\textbf{本题捆绑测试并子任务依赖。}
	
	对于所有数据,$n\le 400,3|n$,$a,b$ 是 $1$ 到 $n$ 的排列。
	\begin{center}
		\begin{tabular}{|C{9em}|C{9em}|C{9em}|C{9em}|}
			\hline
			子任务编号 & 分值 & $n\le $ & 子任务依赖\\ \hline
			1 & $20$ & $10$ & \\ \hline
			2 & $20$ & $20$ & 1\\ \hline
			3 & $10$ & $30$ & 2\\ \hline
			4 & $10$ & $50$ & 3 \\ \hline
			5 & $10$ & $70$ & 4 \\ \hline
			6 & $10$ & $100$ & 5 \\ \hline
			7 & $20$ & $400$ & 6 \\ \hline
		\end{tabular}
	\end{center}
\end{document}